\documentclass[a4paper,11pt,titlepage]{jsarticle}


\usepackage{docmute}
\usepackage{braket}%ブラケット関係のやつ
\usepackage[dvipdfmx]{graphicx}%画像関係のやつ
\usepackage[dvipdfmx]{color}
\usepackage{here}%よくわからん
\usepackage{bm}%ベクトル関係のやつ
\usepackage{amsmath} %数学関係のやつ
\usepackage{listings} %プログラムソースのinclude
\usepackage{color}
% \usepackage{scalefnt}
 
\lstset{ 
  basicstyle={\ttfamily},
  identifierstyle={\small},
  commentstyle={\smallitshape},
  keywordstyle={\small\bfseries},
  ndkeywordstyle={\small},
  stringstyle={\small\ttfamily},
  frame={tb},
  breaklines=true, 
  columns=[l]{fullflexible},
  numbers=left,
  xrightmargin=0zw,
  xleftmargin=3zw,
  numberstyle={\scriptsize},
  stepnumber=1,
  numbersep=1zw,
  lineskip=-0.5ex
} %listingsの設定

\numberwithin{equation}{section} %上手い式の数振り
\setcounter{tocdepth}{3} %subsubsectionまで目次に表示



\begin{document}

    \section{結論}
    \subsection{まとめ}
    集光強度が10$^{20-21}$ W/cm2領域のフェムト秒高強度レーザーを物質に照射することで、
    数Gbarに達する高エネルギー密度プラズマが生成され、プラズマ中にはTV/mに達する強電場やkTオーダの強磁場が形成される。
    これらの強電磁場は、イオンを100 MeV/u領域にまで加速させることが可能であり、
    また高エネルギー粒子を閉じ込める機能を有することから、粒子線癌治療や核融合などの医療・産業への応用、
    さらには高エネルギー宇宙線の生成起源の解明に向けた学術研究への応用などが期待されている。
    一方で、この分野で主に用いられている固体薄膜や希薄ガスをターゲットとして選択した場合、
    これらには電磁場を積極的に保持する機構がないため、ターゲットも音速で崩壊し、
    応用研究の対象はパルス時間程度で起きる現象に限定される。
    我々は、先行する粒子シミュレーション研究により、ターゲットにサブumオーダの微細構造を付与
    (構造性媒質として参照)することで、現在の技術で達成可能なレーザー強度領域で、
    100 TV/mレベルの強電場の生成とそれによる300 MeV領域の準単色陽子線生成や、BGK波の特性を有する
    (レーザーのパルス時間を超えた)準安定非線形波の形成等の多様な現象を見いだしている。
    これを踏まえて本研究では、構造性媒質として直径がサブumオーダの円柱状ケイ素(シリコン)が
    多数配置した物質(ロッド集合体)を選択、相対論的電磁粒子コードEPIC3Dを用いて、
    これに高強度レーザーを照射する2次元(2D)・3次元(3D)シミュレーションを実施し、以下の結果を得た。\\
    \bf
    (1)側面照射(2D)\\
    \rm
    レーザー照射により、ロッド集合体前面に形成される電場構造とイオン加速のメカニズムを詳細に調べた。
    その結果、個々のロッド周囲に形成される電場が重ね合わされることで、準一次元的な強電場構造が形成され、
    この電場により加速されたイオンのエネルギーは、ロッド1本の場合と比較して大きくなることを見いだした。
    また、同一の空間充填率でロッド径を0.125-1.0 µmの領域で変化させた場合、
    イオンのエネルギー(速度)が最大となるロッド径が存在することが分かり、
    ロッド径を調整することで、生成する電場強度やイオンのエネルギーが制御可能であることを新たに見いだした。\\
    \bf
    (2)上面照射(2D)\\
    \rm
    ロッド上面方向からレーザーを照射した場合の、
    ロッド間におけるレーザー場の伝播特性、レーザーエネルギー吸収特性を調べた。
    その結果、レーザー場がTM波としてロッド間を伝播することが分かり、
    ロッド集合体がメタマテリアル様の特性を有することを見いだした。
    また、ロッド間に電流路が形成されることで、電子及びイオンがサブピコ秒の時間スケールで保持されることを明らかにした。
    構造を付与していないシリコン基盤に対し、同一のレーザー条件でシミュレーションを行ったところ、
    同様の結果を得ることはできなかった。
    これは、ロッドの構造を適切に設計することで、電流路およびそれによる磁場強度を制御可能であることを示している。\\
    \bf
    (3)上面照射(3D)\\
    \rm
    結果2を踏まえて、
    理研SACLAで実施されたレーザー照射実験を模擬する3次元シミュレーションを実施した。
    レーザーエネルギーの大部分がロッド集合体により吸収されることが分かり、
    かつ、ロッド各所での電磁波の流入・流失量(ポインティングフラックス)の解析から、
    大部分のエネルギー吸収はロッドの上部の領域で起きていることを明らかにした。
    \subsection{今後の課題}
    上面照射を模擬した3Dシミュレーションでは、高強度レーザー($a_0=2.937$)を用いて1つのケースに関する伝搬解析を行ったが、
    吸収過程においては、レーザー強度、空間充填率など様々なパラメータ依存があると考えられる。
    特にレーザー波長、空間充填率は、ロッド間への伝搬を支配するパラメータであり、
    これらとロッドの吸収特性との関係を明らかにすることは重要な課題と言える。
    また、今回は相対論領域での解析に終始したが、非相対論領域でのシミュレーションでは非線形の効果が弱まり、
    吸収伝搬素過程がより鮮明に確認できると考えられ、ロッドの吸収特性を詳細に理解する手がかりになりえると思われる。

    \newpage
\end{document}

