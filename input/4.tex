\documentclass[a4paper,11pt,titlepage]{jsarticle}


\usepackage{docmute}
\usepackage{braket}%ブラケット関係のやつ
\usepackage[dvipdfmx]{graphicx}%画像関係のやつ
\usepackage[dvipdfmx]{color}
\usepackage{here}%よくわからん
\usepackage{bm}%ベクトル関係のやつ
\usepackage{amsmath} %数学関係のやつ
\usepackage{listings} %プログラムソースのinclude
\usepackage{color}
% \usepackage{scalefnt}
 
\lstset{ 
  basicstyle={\ttfamily},
  identifierstyle={\small},
  commentstyle={\smallitshape},
  keywordstyle={\small\bfseries},
  ndkeywordstyle={\small},
  stringstyle={\small\ttfamily},
  frame={tb},
  breaklines=true, 
  columns=[l]{fullflexible},
  numbers=left,
  xrightmargin=0zw,
  xleftmargin=3zw,
  numberstyle={\scriptsize},
  stepnumber=1,
  numbersep=1zw,
  lineskip=-0.5ex
} %listingsの設定

\numberwithin{equation}{section} %上手い式の数振り
\setcounter{tocdepth}{3} %subsubsectionまで目次に表示


\begin{document}

    \section{ロッド集合体と高強度レーザーの相互作用シミュレーション}
    \subsection{ロッド集合体への側面照射(2D)}
    光と物質との相互作用は物質表面を介して起こるため、比表面積のパラメーター特性を明らかにすることは
    今後の実証実験においても必要不可欠であると言える。
    したがって、本項ではロッドターゲットを特徴付ける要素の1つである比表面積のパラメータ特性について、
    イオン加速の観点から調べることを目的とする。
    過去に行われたシミュレーション結果を図\ref{4-1_single}に示す。
    \begin{figure}[H]
      \begin{center}
        \includegraphics[keepaspectratio,width=\linewidth]{./image/4-1/4-1_single.png}
        \caption{
          \label{4-1_single}
          直径がサブumの単一ロッドと高強度レーザーとの相互作用による、電子・イオンの密度の空間分布図(左)。
          レーザー照射後t=100fs, 140fsの様子を示している。ロッド中心から半径方向(y方向)の密度(電子・イオン)、
          ロッド内外に形成される電場、静電ポテンシャルの1次元図(右)。
        }
      \end{center}
    \end{figure}
    スキン長はロッド径より小さいため、表面の電子のみと相互作用する。
    また、電子はレーザーにより外向きに運動するが、イオンは質量が大きいためレーザーの輻射圧では動かない。
    このとき、静止しているイオンと電子との間に「中心から外向き」に単極の電場が生成される。
    この電場によってイオンは徐々に膨張する(両極性爆発:図\ref{fig:2-4}参照)が、膨張に伴い電場強度は急激に減衰する。
    この結果を踏まえ、今回、実際の実験において作成、使用するシリコンロッド集合体を2Dシミュレーションにて行い再現することで、
    個々のロッドが生成する電場の重ね合わせの効果を確認する。
    \begin{figure}[H]
      \begin{center}
        \includegraphics[keepaspectratio,width=\linewidth]{./image/4-1/4-1_multi.png}
        \label{}
        \caption{
          実際の実験で用いるシリコンロッド集合体の電子顕微鏡写真。
        }
      \end{center}
    \end{figure}
    \bf
    空間充填率\\
    \rm
    空間充填率は単位体積あたりのロッド体積として定義し、これを一定に保つことで比表面積の効果を明らかにする。
    \begin{figure}[H]
      \begin{center}
        \includegraphics[keepaspectratio,width=\linewidth]{./image/4-1/4-1_packing.png}
        \label{}
        \caption{
            空間充填率を一定に保ち、ロッド径$\phi$を$2a$から$a$に変更したときの図。
            ロッド本数$N=4$のときの表面積は、ロッド本数$N=1$のときと比べ2倍の表面積を持つ。
        }
      \end{center}
    \end{figure}
    以降、簡単のため、空間充填率を一定に保ったロッド集合体をロッド径ごとに分類し、
    $\phi=1.0\mu$mのものをA-target、$\phi=0.5\mu$mのものをB-target、
    $\phi=0.25\mu$mのものをC-target、$\phi=0.125\mu$mのものをD-targetと呼ぶ。
    
    \subsubsection{シミュレーションの概要及びシミュレーション条件}
    シミュレーションは図\ref{fig:4-1_rod}に示す通り、x 方向のシステムサイズ $\rm{Lx = 5.12 \mu m}$、y方向のシステムサ
    イズ$\rm{Ly = 15.36 \mu m}$の長方形の領域を設定し、システムの中央部にケイ素ロッドを配置した。
    ケイ素ロッドはA-targetからD-targetまでを比較し、比表面積の違いによるダイナミクスの変化を確認する。
    \begin{figure}[H]
      \begin{center}
        \includegraphics[scale=0.7]{./image/4-1/4-1_rod.png}
        \caption{
          \label{fig:4-1_rod}
            ターゲットの図。空間充填率を一定に保ちながら、ロッド径を変更し、
            比表面積の違いから各系のダイナミクスの変化を確認する。
        }
      \end{center}
    \end{figure}
    この系に対して、パルス幅が40 fs(FWHM)、最大集光強度が$8.0 \times 10^{19} \rm W/cm^2$に達する高強度レーザーを+y方向に照射した。
    ここで、レーザーはy=40 nmに設置したアンテナから誘導電流を流すことにより発振させており、
    x方向には集光せず、強度がx方向に一様の平面波を仮定した。
    また、粒子(電子とイオン)・場(電場・磁場)ともに、xy方向に透過(吸収)境界条件を課している。
    詳細なシミュレーション条件は表\ref{table:4-1}に示す。

    \begin{table}[H]
      
        \caption{
          \label{table:4-1}  
        側面照射を模擬した2Dシミュレーション}
        \centering
        \scalebox{0.8}{
        \begin{tabular}{lc} \hline \hline
            \multicolumn{2}{c}{レーザー条件} \\ \hline
            レーザー強度 & $I=8.0 \times 10^{17} [\rm{W/cm^2}]$ \\
            規格化強度& $a_0 = 0.619$ \\
            波長 & $\lambda_L = 810 [\rm{nm}]$\\ 
            パルス幅  & 40 [fs](FWHM) \\ 
            波形& ガウシアン \\ 
            レーザー空間分布  & 平面波 \\ 
            カットオフ密度& $n_c=1.70\times 10^{21} [\rm{cm^{-3}}]$ \\ \hline
            \multicolumn{2}{c}{ターゲット条件} \\ \hline
            イオン種  & シリコン \\
            電子密度  & $ 6.98 \times 10^{23}[\rm{cm^{-3}}]$ \\
            ロッド直径  & 0.5 [$ \rm \mu$m] \\ \hline
            \multicolumn{2}{c}{系} \\ \hline
            x方向のシステムサイズ& $L_x=2.0 [\rm{\mu m}]$ \\
            y方向のシステムサイズ&$L_y = 15.36 [\rm{\mu m}]$ \\ 
            x方向のメッシュ数  & $1024$ \\
            y方向のメッシュ数  & $1024$ \\
            時間幅  & 6.67 [fs] \\ \hline
            \multicolumn{2}{c}{境界条件} \\ \hline
            x方向粒子  & 周期 \\  
            y方向粒子  & 透過 \\ 
            x方向電磁波  & 周期 \\ 
            y方向電磁波  & 透過 \\ \hline \hline
        \end{tabular}
        }
    \end{table}

    \subsubsection{レーザー場の伝搬特性} 
    図\ref{fig:4-1_absorb_rate}にアンテナからシステム内に入射したエネルギー(赤)、システム内の場(電場・磁場の合計)のエネルギー(青)、
    電子のエネルギー(水色)、イオンのエネルギー(紫)、ポインティング(システムから抜けた電磁場)エネルギー(緑)の時間発展を示す。
    \begin{figure}[H]
        \begin{center}
          \includegraphics[keepaspectratio,width=\linewidth]{./image/4-1/4-1_absorb_rate.png}
          \caption{
            \label{fig:4-1_absorb_rate}
              (a)エネルギーダイナミクスの図。アンテナからシステムに入射したエネルギーを赤、システム内の場のエネルギー(電場・磁場の合計)を青、
              電子のエネルギーを水色、イオンのエネルギーを紫、系外へ流出したエネルギーを緑で示している。
              (b)このときの相対誤差。計算は安定的に解くことができている。
          }
        \end{center}
      \end{figure}
    図から、時刻 t=0 fsよりシステム内の電磁場のエネルギーが上昇し、次点で電子のエネルギーが徐々に上昇していく様子が確認できる。
    これは、レーザー場が質量の大きいイオンを直接動かすことができないことを示している。
    後に電子のエネルギーは、時刻 t=80 fsで最大となり、徐々に減少していく。
    一方、イオンのエネルギーは t=60 fs付近から徐々に上昇しており、
    電子のエネルギーがイオンのエネルギーへと移っていることを示している。
    時刻t=140 fs以降は、ポインティングエネルギーが一定となっている。
    すなわち、システムに対するエネルギーの流入・流出がないことを示しており、
    システム内のエネルギー(電磁場、イオン、電子)のエネルギーが保存していることが分かる。
    このとき、エネルギー吸収率(定常状態におけるシステム内の全エネルギー/アンテナから入射したエネルギー)は 41\%となる。
    \subsubsection{イオン加速のダイナミクス}
    次にロッド内外に形成される電場構造の詳細を図\ref{fig:4-1_density1D}に示す。
    \begin{figure}[H]
      \begin{center}
        \includegraphics[keepaspectratio,width=\linewidth]{./image/4-1/4-1_density1D.png}
        \caption{
          \label{fig:4-1_density1D}
            粒子の個数密度と、電場$E_y$の断面図。x方向に平均化している。横軸は位置($\mu$m)を示す。(a)100 fs時点では電子がロッド表面から剥ぎ取られる様子が確認できる。
            (b)300 fsではイオンが加速し、ターゲット全体が膨張している様子が確認できる。
        }
      \end{center}
    \end{figure}
    レーザー照射後、(a)t=100.05 fs、(b)t=300 fsの時の電場$E_y$(赤)、
    電子密度(青)、イオン密度(黒)についての、y方向の1次元図を示している。
    (a)t=100 fsにおいてロッド後方($y=13 \mu$m付近)で電子、イオン密度が一致しているのに対し、ロッド前面($y=6 \mu$m付近)ではレーザー場により電子が剥ぎ取られている
    ことが確認できる。この結果、電子の密度がイオンの密度よりも小さくなり、荷電分離によりロッドの内外に強電場が形成されることが分かった。
    このときの電場強度はTV/mに達する。(b)t=300 fsではロッドの膨張に伴ってロッド表面の電場が次第に減衰していく様子が分かる。
    一方でイオンは、ロッド内外に残った電場により持続的に加速されるため、時間とともにイオンの最大エネルギーは増大する。
    この時のターゲット前面から飛び出すイオンの速度を、各ターゲットに関して調べたものが以下の図\ref{fig:4-1_17}である。
    \begin{figure}[H]
      \begin{center}
        \includegraphics[keepaspectratio,width=\linewidth]{./image/4-1/4-1_17.png}
        \caption{
          \label{fig:4-1_17}
           (a)ロッド前面のイオン速度。縦軸はロッド速度を光速で割った値、横軸は時間。
           (b)ロッド前面に形成される電場。縦軸は電場のピーク値、横軸は時間。
           ロッド径を変更しても速度の大きな差異は確認できない。
        }
      \end{center}
    \end{figure}
    (a)はイオンの速度に関してのロッド径依存性を調べたものである。どの系においてもイオンの速度に大きな差異は見られない。
    また、この加速の一因であるクーロン力($F=qE$)の観点から、イオン前面に形成される電場のピーク値の時間発展を(b)に示す。
    こちらも大きく差異はなく、速度を示した(a)の図の結果と確かに整合する。

    レーザー強度を一桁上げた$8.0 \times 10^{18}$W/cm$^2$($a_0$=1.96)に設定し、同様のシミュレーションを行った結果が図(\ref{fig:4-1_18})である。
    \begin{figure}[H]
      \begin{center}
        \includegraphics[keepaspectratio,width=\linewidth]{./image/4-1/4-1_18.png}
        \caption{
          \label{fig:4-1_18}
           (a)ロッド前面のイオン速度。縦軸はロッド速度を光速で割った値、横軸は時間。
           (b)ロッド前面に形成される電場。縦軸は電場のピーク値、横軸は時間。
           B-targetとD-targetで約1.3倍ほど速度が異なり、B-targetが最も速い。
           また、(b)から求めたグラフの積分値(力積)に関してもB-taegetが最も大きいことが分かった。
        }
      \end{center}
    \end{figure}
    高強度領域ではロッド径による違いが現れ、B-targetが最も加速されることが確認できた。この時のイオン前面に形成される電場のピーク値の時間発展を(b)に示す。
    各系において、最大値に差異は無いが、その積分値(力積)に違いが現れる。B-targetでは比較的長い時間スケールでロッド前面に形成された強電場が保持されるのに対し、
    D-targetでは電場がすぐに減衰してしまうことが確認できる。これは高強度レーザー照射により、径の小さいロッドが崩壊し、
    1次元的な電場を形成する機構を損なうためであると考えられる。(図\ref{fig:4-1_衝撃波}参照)
    \begin{figure}[H]
      \begin{center}
        \includegraphics[keepaspectratio,width=\linewidth]{./image/4-1/4-1_衝撃波.png}
        \caption{
          \label{fig:4-1_衝撃波}
           ロッド境界面に生成される衝撃波(電場)構造。単一のロッドで確認された電離面での電場が重ね合わせにより増強される。
        }
      \end{center}
    \end{figure}
    \subsubsection{イオン加速と音速}
    ロッドとの相互作用を経て、次第に電子は熱平衡に達し、以降大きな変化は見られなくなる。
    図\ref{fig:4-5_eon}はそのときの電子のエネルギースペクトルである。
    横軸は電子のエネルギー(KeV)で、縦軸は全粒子数で規格化した電子の個数をlogスケールで示している。
    \begin{figure}[H]
      \begin{center}
        \includegraphics[scale=0.8]{./image/4-5_eon.png}
        \caption{
          \label{fig:4-5_eon}
          B-targetのエネルギースペクトル。横軸は電子のエネルギー(KeV)、縦軸は全粒子数で規格化した電子の個数をlogスケールで示している。
          低温側と高温側に2種類のガウス分布を確認することができる。
        }
      \end{center}
    \end{figure}
    この図から電子温度は、低温側と高温側で二温度系になっていることが分かる。
    これはレーザーとの相互作用を経て、エネルギーを受け取ったロッド表面($ \sim \delta$)の電子と、
    レーザーと相互作用をせずに残存するロッドの芯がそれぞれ違う温度であることを示していると考えられる。

    このとき、電子がガウス分布に従うとすると、$E$を電子のエネルギー、$T_e$を電子温度とすると
    \begin{equation}
      F(E) = k e ^{ \frac{E}{T_e} }
    \end{equation}
    すなわち、エネルギースペクトルの傾きから、電子温度を計算することができる。
    また、音速(位相速度)$C_s$は原子番号を$Z$、質量数を$A$、$c$を光速として
    \begin{equation}
      C_s = c \sqrt{\frac{Z T_e}{938 A}}
    \end{equation}
    と表されることから、音速は$ \sim 0.009c $となる。これは図\ref{fig:4-1_17}に示した値の約$1/4$程度である。


    \subsection{ロッド集合体への正面照射(2D)}
    これまでのシミュレーションは側面照射のみにとどまっていたが、ターゲットの持つ特性をより詳細に知るため、
    レーザー照射は多方向から行う必要がある。当研究室ではESM(:)を作成し、電子の特定方向への膨張を実証実験にて確認している。
    
    \begin{figure}[H]
      \begin{center}
        \includegraphics[scale=0.55]{./image/4-2/4-2_ESM.png}
        \label{}
        \caption{
          (a)当研究室で作成したESM()。
          (b)実際の実験における機材のセットアップ。
          レーザー照射は機材の要請からターゲットに対し30°方向に傾け、ESMではレーザーの入射角度に対し垂直な位置での電子を検出する。
          (c)
        }
      \end{center}
    \end{figure}
    一方で、EPIC3Dでは全ての方向でロッドから飛散する粒子を計測しており、実験との比較は容易ではない。
    したがって、今回は従来のシミュレーション結果にポスト処理を加え、
    ESM同様、特定方向での電子計測を行うプログラムを実装し、実験に先立ち様々な角度から膨張を確認する。
    また、これまでのイオン加速を始めとする応用的な視点ではなく、ロッドターゲット間への電磁波の伝搬過程を
    明らかにするため、導波管現象に基づいた基礎的な解析を行う。
    \subsubsection{シミュレーションの概要及びシミュレーション条件}
    シミュレーションは図に示す通り、x方向のシステムサイズ $\rm{Lx = 20.48 \mu m}$、y方向のシステムサ
    イズ$\rm{Ly = 15.36 \mu m}$の領域を設定し、ケイ素ロッド集合体を配置した。
    実際に行われた京大T6レーザー実験を模擬するセットアップにするため、レーザーはターゲットに対し、
    30°傾け照射させる。(図\ref{fig:4-2_rod}参照)
    \begin{figure}[H]
      \begin{center}
        \includegraphics[scale=1.5]{./image/4-2_rod.png}
        \caption{
          \label{fig:4-2_rod}
            ロッド集合体。京大T6レーザーシステムを用いた実験を模擬するため、
            シリコンロッド集合体に対し、30°の角度を付けてレーザーを照射する。
        }
      \end{center}
    \end{figure} 
    詳しいシミュレーション条件は表\ref{table:4-2}に示す。
    \begin{table}[H]
        \caption{
          \label{table:4-2}
          側面照射を模擬した2Dシミュレーション}
        \centering
        \scalebox{0.8}{
        \begin{tabular}{lc} \hline \hline
            \multicolumn{2}{c}{レーザー条件} \\ \hline
            レーザー強度 & $I=1.2 \times 10^{19} [\rm{W/cm^2}]$ \\
            規格化強度& $a_0 = 2.39$ \\
            波長 & $\lambda_L = 810 [\rm{nm}]$\\ 
            パルス幅  & 40 [fs](FWHM) \\ 
            波形& ガウシアン \\ 
            レーザー空間分布  & 集光 \\ 
            カットオフ密度& $n_c=1.70\times 10^{21} [\rm{cm^{-3}}]$ \\ \hline
            \multicolumn{2}{c}{ターゲット条件} \\ \hline
            イオン種  & シリコン \\
            電子密度  & $ 6.98 \times 10^{23}[\rm{cm^{-3}}]$ \\
            ロッド直径  & 0.5 [$ \rm \mu$m] \\ \hline
            \multicolumn{2}{c}{系} \\ \hline
            x方向のシステムサイズ& $L_x=20.48 [\rm{\mu m}]$ \\
            y方向のシステムサイズ&$L_y = 15.36 [\rm{\mu m}]$ \\ 
            x方向のメッシュ数  & $1024$ \\
            y方向のメッシュ数  & $768$ \\
            時間幅  & 6.67 [fs] \\ \hline
            \multicolumn{2}{c}{境界条件} \\ \hline
            x方向粒子  & 透過 \\  
            y方向粒子  & 透過 \\ 
            x方向電磁波  & 透過 \\ 
            y方向電磁波  & 透過 \\ \hline \hline
        \end{tabular}
        }
    \end{table}
    \subsubsection{エネルギー吸収特性}
    図\ref{fig:4-2_absorb_rate}はB-targetにおけるエネルギーの時間発展図である。
    アンテナからシステム内に入射したエネルギー(赤)、システム内の場(電場・磁場の合計)のエネルギー(青)、
    電子のエネルギー(水色)、イオンのエネルギー(紫)、ポインティング(システムから抜けた電磁場)エネルギー(緑)の時間発展を示す。
    \begin{figure}[H]
      \begin{center}
        \includegraphics[keepaspectratio,width=\linewidth]{./image/4-2/4-2_absorb_rate.png}
        \caption{
          \label{fig:4-2_absorb_rate}
            エネルギーダイナミクスの図。アンテナからシステムに入射したエネルギー(赤)、電磁場のエネルギー(青)、
            系内外へと流入・流失したエネルギー(緑)、電子のエネルギー(水色)、イオンのエネルギー(紫)をそれぞれ示す。
            電磁場との相互作用を経て、軽い電子が始めに動き出し、荷電分離が生じて強電場を生成する。
            この強電場によってイオンが加速され、次第にイオンのエネルギーが増え始める。
        }
      \end{center}
    \end{figure} 
    電子のエネルギーは、時刻 t=90 fs付近で最大となり、以降は徐々に減少していく。
    一方、イオンのエネルギーは徐々に増加しており、電子のエネルギーがイオンのエネルギーへと移動していることが確認出来る。
    また、今回の系では電磁場のエネルギーが完全に減衰することはなく、一定の値で下げ止まりになっており、
    シミュレーション時間終端(=400 fs)までシステム内に電磁場が保持(閉じ込め)されていることが分かる。
    これについては後に詳細に議論する。
    今回、シミュレーションの対象とした系は、無衝突プラズマによって構成された準中性の導体極板であるとも考えられる。
    したがって、ここでは導波管現象の基礎理論に基づき、平行極板間を伝播する電磁波について解析した。(図\ref{fig:4-2_TMmode}参照)
    \begin{figure}[H]
      \begin{center}
        \includegraphics[scale=0.55]{./image/4-2/4-2_TMmode.png}
        \caption{
          \label{fig:4-2_TMmode}
          ロッドターゲット間を伝搬する電磁波をフーリエ変換したもの。
          特定の波数(モード)がロッド間で確認された。
        }
      \end{center}
    \end{figure} 
    この図から、特定のモードをTM波として伝搬していることが確認され、入射角度を最適化することで、より吸収率の高い
    パラメータを見つけ出すことができると考えられる。

    \subsubsection{ESMを用いた実験との比較}
    電子の位相図を図\ref{fig:4-2_dis}(a)に示す。
    \begin{figure}[H]
      \begin{center}
        \includegraphics[scale=0.55]{./image/4-2/4-2_dis.png}
        \caption{
          \label{fig:4-2_dis}
            (a)電子の位相図。縦軸はy方向の運動量を電子の運動量で規格化したもの、横軸はx方向の運動量を電子の運動量で規格化したものである。
            ロッド電子の全ての運動量分布を示している。
            (b)(a)をスペクトル化したもの。横軸はエネルギー、縦軸は個数を電子数で規格化したものである。
            滑らかな分布を持ちマクスウェル分布として近似できる。
        }
      \end{center}
    \end{figure} 
    縦軸はy方向の運動量を電子の運動量で規格化したもの、横軸はx方向の運動量を電子の運動量で規格化したものである。
    またこの時の電子スペクトルが図\ref{fig:4-2_dis}(b)である。
    ロッドから飛散する全ての粒子の情報をこの図から得ることができるが、実際の実験ではレーザーの入射方向に対し、
    垂直な直線上にESMを設置するため、このような結果を得ることはできない。
    したがって、レーザー入射方向に対し90°±5°のデータだけを抽出し、これをスペクトルに変換することで
    ESMから得られるデータを再現した。(図\ref{fig:4-2_30}\ref{fig:4-2_0})
    \begin{figure}[H]
      \begin{center}
        \includegraphics[scale=0.55]{./image/4-2/4-2_30.png}
        \caption{
          \label{fig:4-2_30}
            (a)データ処理後の位相図。ESMの設置方向(レーザーに対して90°の位置)に飛散する粒子だけを捕捉している。
            (b)(a)を変換して作成したエネルギースペクトル。
        }
      \end{center}
    \end{figure} 
    比較のため、レーザーに対して120°の方向にも同様にESMを設置したものが図\ref{fig:4-2_0}である。
    \begin{figure}[H]
      \begin{center}
        \includegraphics[scale=0.55]{./image/4-2/4-2_0.png}
        \caption{
          \label{fig:4-2_0}
            (a)データ処理後の位相図。ESMの設置方向(レーザーに対して90°の位置)に飛散する粒子だけを捕捉している。
            (b)(a)を変換して作成したエネルギースペクトル。
        }
      \end{center}
    \end{figure} 
    方向別にスペクトルを出力し、ESMのデータを再現することが出来たが、
    一方で比較に足るほど十分な粒子数がスペクトルから確認できないことが分かる。
    これは、シミュレーション時に系を満たす粒子数を増やすことで改善できると考えられる。

    \subsubsection{準定常磁場の生成}
    前述のとおり、系内には長時間に渡り、電磁場が系内に滞留していることが図(エネルギーダイナミクス)から分かった。
    このときの電磁場の詳細な内訳を以下の図\ref{fig:4-3_field}に示す。
    \begin{figure}[H]
      \begin{minipage}{0.45\hsize}
       \begin{center}
        \includegraphics[keepaspectratio,width=\linewidth]{./image/4-3_field.png}
       \end{center}
       \caption{
        \label{fig:4-3_field}
        電磁場の各成分。横軸は時間[fs]で、縦軸は電磁場の強度を示す。
       ターゲットにレーザーが到達するまでの時間では、真空を伝搬する電磁波として、$E_x$と$B_z$の強度が完全に一致する。($\sim $50 fs付近)
       以降はイオン前面にできる電場$E_y$が大きくなり次第に減衰する。シミュレーション終端ではBzだけが一定の強度を保持する。}
       
      \end{minipage}
      \hfill
      \begin{minipage}{0.5\hsize}
       \begin{center}
        \includegraphics[keepaspectratio,width=\linewidth]{./image/4-2/4-2_Bz2D.png}
       \end{center}
       \caption{
        \label{fig:4-2_Bz2D}  
       (a)磁場Bzの2D平面図(30°入射)。(b)レーザーを正面から入射させた系(0°入射)。どちらもロッド間に400fsもの間磁場が保持されている。
       また、0°入射では磁場配位が左右に対称性を持ち、この配位はレーザーの入射角度に起因するものであることが分かる。}
      \end{minipage}
 \end{figure}
    図\ref{fig:4-3_field}から、ターゲットにレーザーが到達するまでの時間では、真空を伝搬する電磁波として、$E_x$と$B_z$の強度が完全に一致することが確認できる。
    ($\sim $50 fs付近)後にレーザーとの相互作用を経て、側面照射シミュレーション同様、次第にシース電場$E_y$がイオン前面に形成される。
    150 fs以降では磁場$B_z$だけが一定の強度を保ち、残りの電磁波成分は大きく減衰する。
    また、図\ref{fig:4-2_Bz2D}から、系内に滞留する磁場はロッド間に保持されており、この時の磁場配位はレーザーの入射角度に依存していることが分かった。
    
    次に、基盤だけの系を用いて同様の解析を行ったが、このような準定常な磁場は確認できず、
    これがロッドターゲットを対象にしたレーザー照射でのみ起こる特有の現象であることが確認された。(図\ref{fig:4-2_Bz}参照)
    \begin{figure}[H]
      \begin{center}
        \includegraphics[scale=0.9]{./image/4-2/4-2_Bz.png}
        \caption{
          \label{fig:4-2_Bz}
            基盤のみでのシミュレーション結果との比較。このような系においては磁場が保持されず、急激に
            減少してしまう。
        }
      \end{center}
    \end{figure} 
    磁場を保持する機構を探るため、電流源を確認する。磁場配位は図\ref{fig:4-2_Bz}に示すとおり特徴的な配位であり、
    マクスウェル方程式から、これを積極的に保持する電流の駆動が示唆される。
    磁場が保持されているt=400 fsにおける電流$J_y$は以下のとおりである。
    \begin{figure}[H]
      \begin{center}
        \includegraphics[scale=0.6]{./image/4-3_currentave.png}
        \label{}
        \caption{
            電流Jyをロッド方向に平均したもの。
        }
      \end{center}
    \end{figure} 
    y方向に流れる電流は、ロッドの表面付近で左右対象に駆動していることが確認できる。
    一方で、レーザーとの相互作用を経てロッド表面には電子の圧力勾配が生まれる。
    \begin{figure}[H]
      \begin{center}
        \includegraphics[scale=0.3]{./image/4-3_pressure.png}
        \label{}
        \caption{
            圧力勾配
        }
      \end{center}
    \end{figure} 
    ここまでを模式的に示したものが図(\ref{})である。
    \begin{figure}[H]
      \begin{center}
        \includegraphics[scale=0.7]{./image/4-3_current.png}
        \label{}
        \caption{
            電流
        }
      \end{center}
    \end{figure} 
    このとき$\bm{B} \times \nabla \bm{P_e}$による電流が駆動され、この直線電流により磁場が保持されている。

    \subsection{ロッド集合体への正面照射(3D)}
    本節では、これまでの2Dシミュレーションでは厳密に扱うことのできなかった、高さ方向のパラメータ特性について議論する。
    具体的な方法としては、空間充填率を一定に保った従来のB-target(\ref{})に加え、この高さを半分の2.5$ \mu$mとしたB'-targetとの比較を行う。
    構造性媒質はターゲット表面を介して場と粒子が相互作用するため、高さを半分にすると比表面積は半分となり、
    エネルギー吸収の特性に違いが現れると考えられる。これを起点にロッドの持つ吸収特性について明らかにしていく。
    \subsubsection{シミュレーションの概要及びシミュレーション条件}
    シミュレーションは図(\ref{})に示す通り、x 方向のシステムサイズ $\rm{Lx = 1.0 \mu m}$、y方向のシステムサ
    イズ$\rm{Ly = \mu m}$、$\rm{Lz = 1.0 \mu m}$の三次元領域を設定し、システムの中央部にケイ素ロッドを配置した。
    ケイ素ロッドのパラメータ特性を調べるため、ロッド径を固定し高さを半分(=2.5 $\mu$m)にしたものと比較する。
    構造性媒質と電磁波との相互作用は表面を介して起こるため、表面積の比からB'ターゲットのロッド粒子1個辺りが受け取るエネルギーは
    Bターゲットのおおよそ2倍になると予想される。

    この系に対して、パルス幅が40 fs(FWHM)、最大集光強度が$1.2 \times 10^{19} \rm W/cm^2$に達する高強度レーザーを+y方向に照射した。
    ここで、レーザーはy=40 nmに設置したアンテナから誘導電流を流すことにより発振させており、
    x方向には集光せず、強度がx方向に一様の平面波を仮定した。
    また、粒子(電子とイオン)・場(電場・磁場)ともに、xy方向に透過(吸収)境界条件を課しているため、
    システム内にはロッドが一本であるが、x、z方向は一様にロッドが整列した実験と同様の設定を再現している。
    詳細な設定は表\ref{table:3D}に示す。


    
    \begin{table}[H]
      \caption{正面を模擬した3Dシミュレーション}
      \centering
      \label{table:3D}
      \scalebox{0.8}{
      \begin{tabular}{lc} \hline \hline
          \multicolumn{2}{c}{レーザー条件} \\ \hline
          レーザー強度 & $I=8.0 \times 10^{19} [\rm{W/cm^2}]$ \\
          規格化強度& $a_0 = 0.61$ \\
          波長 & $\lambda_L = 810 [\rm{nm}]$\\ 
          パルス幅  & 40 [fs](FWHM) \\ 
          波形& ガウシアン \\ 
          レーザー空間分布  & 平面波 \\ 
          カットオフ密度& $n_c=1.70\times 10^{21} [\rm{cm^{-3}}]$ \\ \hline
          \multicolumn{2}{c}{ターゲット条件} \\ \hline
          イオン種  & シリコン \\
          電子密度  & $ \times 10^{22}[\rm{cm^{-3}}]$ \\
          ロッド直径  & 0.5 [$ \rm \mu$m] \\ \hline
          \multicolumn{2}{c}{系} \\ \hline
          x方向のシステムサイズ& $L_x=2.0 [\rm{\mu m}]$ \\
          y方向のシステムサイズ&$L_y = 15.36 [\rm{\mu m}]$ \\ 
          x方向のメッシュ数  & $1024$ \\
          y方向のメッシュ数  & $1024$ \\
          時間幅  & 6.67 [fs] \\ \hline
          \multicolumn{2}{c}{境界条件} \\ \hline
          x方向粒子  & 周期 \\  
          y方向粒子  & 透過 \\ 
          x方向電磁波  & 周期 \\ 
          y方向電磁波  & 透過 \\ \hline \hline
      \end{tabular}
      }
    \end{table}
    
    
    \subsubsection{エネルギー吸収特性}
    図\ref{fig:4-3_absorb_rate}は左図が$\phi = 5.0 \mu \rm{m}$、右図(\ref{})が$\phi = 2.5 \mu \rm{m}$の系におけるエネルギーの時間発展図である。
    縦軸はlogスケールにしている。アンテナからシステム内に入射したエネルギー(赤)、システム内の場(電場・磁場の合計)のエネルギー(青)、
    電子のエネルギー(水色)、イオンのエネルギー(紫)、ポインティング(システムから抜けた電磁場)エネルギー(緑)の時間発展を示す。
    基盤とロッドのエネルギーはそれぞれ実線と点線で示している。
    \begin{figure}[H]
      \begin{center}
        \includegraphics[keepaspectratio,width=\linewidth]{./image/4-3/4-3_absorb_rate.png}
        \caption{
          \label{fig:4-3_absorb_rate}
          (a)B-targetと(b)B'-targetのエネルギーダイナミクス。電子、イオンが受け取るエネルギーに差異は無く、
          また基盤が受け取るエネルギー(点線)の寄与が少ないことから、ロッドの高さ方向に関する吸収特性の存在が示唆される。
        }
      \end{center}
    \end{figure}
    図から分かるとおり、基盤はほとんどエネルギーを受け取っておらず、エネルギー吸収の大部分はロッドによるものであることが分かる。
    また、各系における粒子のエネルギーは、シミュレーション時間終端(t=200 fs)において同程度となっている。
    この結果から、ロッド間への電磁波の伝搬は一様なものではなく、
    ロッドの高さ方向に関して何らかの特徴がある可能性が考えられる。

    例えば、真空の系(図\ref{fig:4-3_vacuum2})において、200 fs間で各点を通過する電磁波のエネルギーの積分値は等しくなるが、
    一方ターゲットを配置した場合には、粒子との相互作用を経てエネルギーの交換が起こる。   
    \begin{figure}[H]
      \begin{center}
        \includegraphics[keepaspectratio,width=\linewidth]{./image/4-3/4-3_vacuum2.png}
        \caption{
          \label{fig:4-3_vacuum2}
            真空の系。電磁波は減衰せず+y方向へと進行し、各面で評価したパルスの積分値はそれぞれ等しくなる。
            (a)は面1上でのレーザーパルス(b)は面2上でのレーザーパルスを指す。真空の系では
            面1を通過したレーザーパルスはそのまま減衰せず面2を通過する。
        }
      \end{center}
    \end{figure}
    系全体のエネルギー吸収率は図\ref{}より、B-target、B'-target共に90\%を超え、また、ロッドでのエネルギー吸収が支配的であることから、
    ロッド上の各所でポインティングフラックスを評価し(図\ref{fig:4-3_rod}参照)、
    各面を通過する電磁波の値と、その減衰率(dumping rate)を求めることで
    ロッドターゲットのエネルギー吸収特性を調べることができる。(\ref{})はその結果である。
    \begin{figure}[H]
      \begin{center}
        \includegraphics[scale=0.6]{./image/4-3/4-3_rod.png}
        \caption{
          \label{fig:4-3_rod}
            ロッドを配置した系。各面でのポインティングフラックスから減衰率を計算し、
            ロッドの高さ方向に関する吸収率及び伝搬過程を調べる。
        }
      \end{center}
    \end{figure}
    \begin{figure}[H]
      \begin{center}
        \includegraphics[scale=0.5]{./image/4-3/4-3_absorb.png}
        \label{}
        \caption{
            ロッド上の各点におけるpoynting flux。縦軸はエネルギー[V/m$^3$]、横軸はロッドの位置座標を示す。
            B-target(緑):入射した電磁波は、ロッド中腹に到達するまでに大きく減衰し、エネルギーは電子、イオンに移っていることが分かる。
            高さ2.5 $\mu$mの位置まで80\%近くの吸収が起こっていることが分かる。
            B'-target(青):2.5$\mu$m進むまでに入射エネルギーの80\%ほどがロッドに吸収されている。
        }
      \end{center}
    \end{figure}
    B-targetに関して、電磁波はロッド先端(y=0)の位置から次第に減衰し、粒子にエネルギーを受け渡し始める。
    電磁波がおよそ2.5 $\mu$m進むまでにロッド先端位置での入射エネルギーに対して80\%程度の電磁波のエネルギーが
    減衰し、粒子に吸収される。ロッドの中心位置(y=2.5$\mu$m)以降では残りのエネルギーが少しずつ吸収されていき、
    ロッドの根本付近まで電磁波はほぼ到達していない。
    B'-targetでは、電磁波がロッドに到達するまで(y=2.5$\mu$m)ほぼ一定の値となることが確認できる。
    このとき、真空における電磁波の伝搬と比べ過小にポインティングが評価されているが、
    これはシミュレーション領域が小さく、電磁波が入射した直後から
    粒子へのエネルギー伝搬が始まっていることが原因であると考えられる。
    ロッドの入り口に電磁波が到達するとポインティングは急激に減衰し始め、電磁場のエネルギーが粒子へと移動していく。
    こちらも同様にロッドの根本付近にはほぼ到達していない。

    一方で、反射波の影響が大きい場合、各面で計算したポインティングが過小に評価される恐れがあり、
    このとき、減衰率をそのまま各位置での吸収率とみなすことはできない。
    したがってB-targetから反射の要因となりえる基盤を取り除き、これらの結果を比較することで反射波の影響について調べた。
    図\ref{fig:4-3_absorb2}はその結果である。
    \begin{figure}[H]
      \begin{center}
        \includegraphics[scale=0.5]{./image/4-3/4-3_absorb2.png}
        \caption{
          \label{fig:4-3_absorb2}
           基盤を取り除いた系(黒)と、基盤を入れた従来の系(緑)の減衰率に差異は見られず、
           基盤によって反射された電磁波の影響は小さいと見られる。
           また、基盤を取り除いた場合、y=5$\mu$m以降の座標ではポインティングの増減が見られない。
           これは、ロッド間を伝搬する過程で電磁波が全て吸収されていることを示している。
        }
      \end{center}
    \end{figure}
    色線がB-target、色線がB-targetから基盤を取り除いた系で、これらの結果に差異は無く、
    基盤による反射波の影響は無視できると考えられる。

    ロッド間ではレーザーによって真空領域が時々刻々プラズマで満たされていく。
    このプラズマによる反射により上面付近での吸収が支配的になると考えられる。
    % \subsubsection{電子温度・イオン温度のロッド径依存性}
    % 次に空間充填率を一定に保ち、ロッド径を変えた場合の電子温度について考察する。

    \newpage
\end{document}

