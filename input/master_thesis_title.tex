
\thispagestyle{empty}

\begin{center}
  
  \vspace*{3zw}

  {\LARGE{修士学位論文}}

  \vspace{3zw}

\end{center}

\begin{flushleft}


\end{flushleft}

\begin{center}

  {\huge{{高強度レーザーに照射された構造性媒質中での}}}

  {\huge{{電磁波の伝搬・エネルギー吸収特性に関する研究}}}

  \vspace{3zw}

  \LARGE{京都大学大学院エネルギー科学研究科}

  \LARGE{エネルギー基礎科学専攻 修士課程}

  \vspace{3zw}

  \LARGE{{上田\hspace{1zw}永樹}}

  \vspace{4zw}

  \LARGE{指導教員名}
  
  \Large{岸本\hspace{1zw}泰明\hspace{1zw}教授}



  \vspace{4zw}

  \LARGE{提出日}

  \LARGE{令和4年1月31日}

\end{center}


% \vspace*{-\large{2zw}
% \thispagestyle{empty}
% \begin{abstract}
%   \vspace{-1zw}
%   \normalsize{実験室や宇宙などにおいて状態の異なるプラズマの接触面近傍で形成される無衝突プラズマ境界層では、様々なプラズマ波動が関わる多彩な非線形現象が創出される。中でも星間ガス中での超新星爆発に伴う無衝突衝撃波は、プラズマ波動の一つとして、高エネルギー宇宙線の生成メカニズムの有力候補と考えられており、高強度レーザーを用いた実験室宇宙物理における研究の対象の一つとなっている。この動機付けのもと、本研究では、相対論的電磁粒子コードEPIC3Dを用いて、空間に局在した固体物質と背景ガスからなるターゲットと高強度レーザーとの相互作用を模擬する一次元粒子シミュレーションを実施し、境界層を持つプラズマを形成し、異なる二つの媒質を介して生じる無衝突衝撃波の伝播・維持特性を調べた。ターゲットは、電子密度10$n_{c}$($n_{c}$:遮断密度)のケイ素スラブと電子密度0.1 $n_{c}$の炭素ガスを接触させて配置した系を選択し、レーザーパラメータは、大阪大学LFEXレーザーおよび量研関西研J-KAREN-Pレーザーを想定し、波長0.81 $\mu$m、最大集光強度が$1.0\times10^{19} (\rm{W/cm^2})$として、パルス幅 40 fs(FWHM)で起ち上がり後、強度一定の連続波とパルス長40fsのパルス波をそれぞれ用いた。

%   高強度レーザー(連続波)を固体(ケイ素スラブ)側に照射すると、レーザー光の動重力により固体表面の電子がレーザー伝播方向(+y方向)に加速され、固体裏面のおいてシース電場が形成される。この電場により、ガスイオンは圧縮されて、急峻な密度勾配を持つ構造(ガスの圧縮面)が形成される。この構造は、時間の経過とともに、無衝突衝撃波として上流のガスイオンを取り込みながら+y方向に伝播し、下流領域を形成する。このとき、ガスの圧縮面がイオン過多となることから、その近傍にはTV/mに達する強電場が生成する。この電場が形成する静電ポテンシャルは、ガスの圧縮面(衝撃波面)の伝播とともに+y方向に運動し、上流のガスイオンは反射・加速される(衝撃波加速)。このとき、連続波により固体側から供給された高エネルギー電子が、衝撃波面近傍に形成される静電ポテンシャルにより反射されることで、衝撃波を形成する電場にエネルギーを付与する。そのエネルギーは、衝撃波が上流ガスイオンを反射し加速することによる散逸と、高エネルギー電子による供給がバランスすることで、準定常状態となる。これにより、衝撃波は減衰することなく、長時間($~$ps)にわたりその構造を保持することが明らかとなった。
%   極短パルスレーザーを照射した場合、初期においては、連続波照射の場合と同様にシース電場により形成されたガスの圧縮面が+y方向に伝播するが、レーザーによるエネルギー供給がパルス長程度と短く、ガスイオンの電荷密度分布は衝撃波構造に成長せず、孤立波となる。一方で、固体イオンとガスイオンの電荷密度の和をとると、衝撃波構造が形成されており、さらにガスにおいて上流イオンの反射が生じていることがわかった。このことから、パルス入射の場合は、固体側から供給された膨張イオンが、ガスの圧縮面における電場により反射されることで、衝撃波を形成する電場にエネルギーを供給していることを見いだした。}
  


% \end{abstract}



